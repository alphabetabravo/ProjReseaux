%\thispagestyle{fancy}

\section*{Remerciements}
Tout d'abord, j'exprime ma profonde gratitude envers Javier SERRANO pour m'avoir pris en stage, ainsi qu'� Philippe NOUCHI sans qui il aurait �t� plus dur d'entrer au CERN. Bien s�r, je les remercie �galement pour leurs nombreux conseils. Ils m'ont permis d'organiser mon travail d'un point de vue professionnel. 

Je me dois �galement de remercier Nicolas DE METZ-NOBLAT, un administrateur qui s'est occup� de la r�installation de mon syst�me et Christine GAYRAUD pour le suivit des diff�rentes commandes.

J'adresse aussi mes remerciements aux autres coll�gues pour la bonne ambiance et les quelques conseils toujours bienvenus. Plus particuli�rement � Matthieu CATTIN et Olivier BARRI\` ERE pour la r�alisation du prototype.

C'est un r�el plaisir de travailler dans une telle �quipe.


\section*{Note}
\diamondpar{Ce projet est r�alis� � base de Pic 18F2550 mais peut facilement �tre adapt� pour d'autres microcontr�leurs. De plus, ce rapport a �t� r�dig� en \LaTeX\ en accord avec le monde du logiciel libre. Ce langage �tant compil�, il permet �galement une mise en page de qualit� et normalis�e. Je vous recommande d'ailleurs la classe Beamer de \LaTeX\ permettant de r�aliser rapidement des pr�sentations et des transparents de tr�s grandes qualit�s. \og La connaissance s'acquiert par l'exp�rience, tout le reste n'est que de l'information\fg\ - Albert Einstein}


