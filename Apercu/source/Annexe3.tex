\thispagestyle{empty}

\chapter{Cartes}

\section{Emetteur}
Vous trouverez page suivante le sch�ma de la carte �mettrice. Celle qui mesure la temp�rature. Pour un soucis de consommation, plusieurs r�gulateurs avec "Enable" ont �t� utilis�s pour pouvoir couper l'alimentation de tout ce qui est inutile � certain moment.

Puis les pages suivantes sont directement tir�e du fichier PostScript export� par Pcb. Il s'agit des typons de la carte �mettrice. 

\section{R�cepteur}
Ensuite vous trouverez le sch�ma de la carte r�cepteur. Bien remarquer le condensateur de 470~nF. C'est par lui qu'est fix� le potentiel 3,3~V pour fixer la vitesse de transmission sur la liaison USB. 

Puis les typons de la carte r�cepteur export�s par Pcb. 3 feuilles parmis 12 (masque, trou de per�age, \dots)

\begin{landscape}
%%\addcontentsline{toc}{chapter}{Sch�ma}
\begin{figure}[h]
    \includegraphics[]{./img/Emet}
    \caption{Schema \' Emetteur}
\end{figure}
\end{landscape}

\includepdf[pages={2,3,11},pagecommand={\thispagestyle{plain}}]{Emetteur} 



\begin{landscape}
\begin{figure}[]
    \includegraphics[width=25cm]{./img/Recep}
    \caption{Schema \' Emetteur}
\end{figure}
\end{landscape}
\includepdf[pages={2,3,11},pagecommand={\thispagestyle{plain}}]{Recepteur} 


%\includepdf[pages={{},1},nup=1x2,landscape=true]{output}
%\includepdf[pages={{},2},nup=1x2,landscape=true,angle=180]{output}
%\includepdf[pages={6,3},nup=1x2,landscape=true]{output}
%\includepdf[pages={5,4},nup=1x2,landscape=true,angle=180]{output} 


%\chapter{Organigramme}
%\addcontentsline{toc}{chapter}{D~~Organigramme}
%\includepdf[pages={1},nup=1x1,landscape=true]{AB_CO} %% nup pour l'echelle
