\chapter*{Pr�sentation}
\addcontentsline{toc}{chapter}{Introduction}
\section*{Nom du projet}
Nous avons d�cid� de nommer notre projet "MamieWifi". Le nom scientifique du projet est : Rogue Access Point.
\newline
Notre choix pour ce nom est enti�rement tir� du but final du projet. 

\section*{Objectifs du projet}
Le principal objectif de ce projet est d'associer plusieurs t�chnologies de l'informatique et des r�seaux pour mettre en ouevre une attaque de type "man in the middle" sur un r�seau wifi.
\newline

Ce projet consiste � usurper l'identit� d'une borne wifi pour que le traffic r�seau des utilisateurs connect�s soit automatiquement redirig� vers notre borne wifi corrompu.
\newline
\newline
Afin d'atteindre cet objectif nous allons proc�der ainsi :
\begin{itemize}\renewcommand{\labelitemi}{$\bullet$}
\vspace{3mm}
\item
Creation d'un point d'acc�s � l'aide de l'outil Airbase-ng
\item
Mise en place d'un serveur DHCP pour les clients
\item
Mise en place d'un serveur DNS pour rediriger certaines demandes (Facebook, maBanque, Gmail...) vers un site pi�g�.
\item
Mise en place d'une proc�dure ("Hame�onnage") pour r�cup�rer la cl� wifi de la vrai borne wifi
\end{itemize}
\section*{Nouvelles comp�tences vis�es}
Ce projet � pour but de nous faire comprendre l'installation, la configuration et l'analyse des outils cit�s pr�c�demment.
\newline
\newline
Voici les notions qui serons abord�es pour ce projet :
\begin{itemize}\renewcommand{\labelitemi}{$\bullet$}
\vspace{3mm}
\item
Fonctionnement d'une borne wifi ( Analyse des trames avec wireshark )
\item
Utilisation des outils de la suite aircrack-ng
\item
Analyse du fonctionnement d'un serveur DHCP
\item
Analyse du fonctionnement d'un serveur DNS
\item
Developpement de plusieurs sites web (Apache / SQL)
\item
Analyse du SSL Stripping
\end{itemize}
\newpage
\section*{Organisation}
Pour l'organisation de notre travail nous avons pris la d�cision de travailler de la mani�re suivante :
\begin{itemize}\renewcommand{\labelitemi}{$\bullet$}
\vspace{3mm}
\item
Partage des documents, fichiers de configuration, ... via github
\item
Communication par t�l�phone et email
\item
Deux cr�naux de 2h par semaine pour travailler tous ensembles
\end{itemize}

