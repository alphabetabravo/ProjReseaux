\documentclass[hyperref={bookmarks=false}]{beamer}
\usepackage[latin1]{inputenc}
\usepackage{verbatim}
\usepackage{aeguill}
\usepackage{amsfonts}
\usepackage{xspace}
\usepackage{amsmath}
\usepackage{textcomp} 
\usepackage{cancel} 
\usepackage{mathcomp} 
\usepackage{makeidx}
\usepackage{lscape} % pour avoir une page en paysage
\usepackage{fancybox}
\usepackage{shapepar}	% pour les mises en forme de texte (losange etc)
\usepackage{fancyhdr}
\usepackage{textcomp}  % pour les caract�res sp�ciaux (copyleft etc)
\usepackage{marvosym} % caractere electrostatic, laserbeam...
\usepackage{supertabular}% tableaux sur plusieurs pages
\usepackage{ifsym} % caract�res scp�ciaux
\usepackage{longtable}
\usepackage{listings}
\usepackage{multicol}
\usepackage{upgreek}
%\usepackage{picins}
%\usepackage{frbib}
%\usepackage{lastpage}
\usepackage[english,french]{babel}
%%%%%%%%%%%%% parallel, multicolpar et parcolumns.%%%%%%%%%%%%%%%%%%%%%% pour des documents multicolones bilingue.
\usepackage{graphicx, color}
\usepackage{pdfpages}
\graphicspath{{img/}} % La ou j'ai mes images
\DeclareGraphicsExtensions{.jpg,.png}
\pdfoutput=1
%\usepackage[pdftex,
%	a4paper,
%    bookmarks = true,           % Signets
%    bookmarksnumbered = true,   % Signets numerotes
%    pdfpagemode = None,         % Signets/vignettes fermes a l'ouverture
%    %pdfstartview = FitH,        % La page prend toute la largeur
%	pdfstartview = FitV,        % La page prend toute la hauteur
%	colorlinks=true,
%	citecolor=black,urlcolor=blue,linkcolor=red,
%	pdfauthor={R�mi Mollard},
%	pdftitle={overview reseaux},
% 	pdfsubject={rogueAp},
%	pdfkeywords={},
%	plainpages=false,
%	pdfpagelabels,
%	bookmarks,
%	breaklinks=true,
%   	hyperindex,
%	linktocpage=true
%]{hyperref}
\pdfcompresslevel=9
%
%
%%% style sans serif pour les url
\urlstyle{sf}





%D�finition du th�me
\usetheme{default}

% Define some colors:
\definecolor{DarkFern}{HTML}{407428}
\definecolor{DarkCharcoal}{HTML}{4D4944}
\colorlet{Fern}{DarkFern!85!white}
\colorlet{Charcoal}{DarkCharcoal!85!white}
\colorlet{LightCharcoal}{Charcoal!50!white}
\colorlet{AlertColor}{orange!80!black}
\colorlet{DarkRed}{red!70!black}
\colorlet{DarkBlue}{blue!70!black}
\colorlet{DarkGreen}{green!70!black}

% Use the colors:
\setbeamercolor{title}{fg=Fern}
\setbeamercolor{frametitle}{fg=Fern}
\setbeamercolor{normal text}{fg=Charcoal}
\setbeamercolor{block title}{fg=black,bg=Fern!25!white}
\setbeamercolor{block body}{fg=black,bg=Fern!25!white}
\setbeamercolor{alerted text}{fg=AlertColor}
\setbeamercolor{itemize item}{fg=Charcoal}


\begin{document}

\begin{frame}
		\titlepage
        \title{Installation et configuration d'un rogue access point}
        \author{Adrien Cou�ron, J�r�mie Liebhardt, R�mi Mollard, Thomas Gambart}
        \institute{ENSIBS - Cyberd�fense}
        \date{Lundi 21 septembre 2015}
        \logo{}
        %\includegraphics[height=10mm]{logo-ensibs-horizontal.jpg}
\end{frame}

\begin{frame}
		\tableofcontents[pausesections]
\end{frame}

\begin{frame}
		\tableofcontents[hideothersubsections]
\end{frame}

\section{Objectifs pr�vus}

\begin{frame}
		\tableofcontents[hideothersubsections]
\end{frame}

\begin{frame}
        \frametitle{Objectifs pr�vus dans notre Overview}
        Cr�ation d'un point d'acc�s � l'aide de l'outil airbase-ng
		Mise en place d'un serveur DHCP pour les clients
		Mise en place d'un serveur DNS pour rediriger certaines demandes (Facebook,maBanque, Gmail...) vers un site pi�g�.
		Mise en place d'une proc�dure d'Hame�onnage pour r�cup�rer la cl� Wi-Fi de la vrai borne Wi-Fi cibl�
\end{frame}


\section{Objectifs non atteints}


\begin{frame}
		\tableofcontents[hideothersubsections]
\end{frame}




\section{Objectifs atteints}

\begin{frame}
		\tableofcontents[hideothersubsections]
\end{frame}


\subsection{Fake base en utilisant airebase-ng}
\begin{frame}
        \frametitle{Fake AP en utilisant airebase-ng}
\end{frame}

\subsection{Rogue AP}
\begin{frame}
        \frametitle{Rogue AP}
        Configuration du hotspot avec hostap(d?)
        	Cr�ation d'un hotspot avec la m�me s�curit� que l'AP cibl�
        	Une fois le client connect�, identification gr�ce au WPS
        	Une fois le client connect�, il reste connect� sur la borne m�me si l'attaque s'arr�te
        
        Limite de la solution :
        	La connexion doit �tre initi� par l'ordinateur cible
\end{frame}

\subsection{Int�gration routage internet}
\begin{frame}
        \frametitle{Int�gration routage internet}
\end{frame}

\subsection{Man In The Middle}
\begin{frame}
        \frametitle{Man In The Middle}
\end{frame}

\subsection{Faux sites}
\begin{frame}
        \frametitle{Mise en place de faux site}
\end{frame}

\subsection{Keylogger en javascript et python}
\begin{frame}
        \frametitle{Keylogger en javascript et python}
\end{frame}

\subsection{Extension aka chose possible}
\begin{frame}
        \frametitle{Extension aka chose possible}
\end{frame}



\section{L�galit�}
\begin{frame}
        \frametitle{L�galit� des op�rations}
        cadre et limites
        Comment en respecter le cadre l�gal ?
\end{frame}


\section{Conclusion}
\begin{frame}
        \frametitle{Conclusion}
\end{frame}

\section{Questions ?}
\begin{frame}
        \frametitle{Merci de votre �coute}
        Nous sommes ouverts � toutes vos questions.
\end{frame}

\end{document}
